% !TEX encoding = UTF-8 Unicode
\documentclass{article}

\usepackage{multirow}
\usepackage[bottom]{footmisc}
\usepackage{url}
\usepackage{hyperref}
\usepackage{xepersian}
\usepackage{ulem}


\settextfont{XB Zar}

\renewcommand{\footnoterule}{
	\hrule width 3in
}
\rightfootnoterule

\title{\textbf{گام اول پروژه درس تحلیل و طراحی سیستم‌ها}}

\author{سید پارسا میرطاهری - ۹۵۱۰۹۳۹۴ \\ کیارش گل‌زردی - ۹۵۱۰۵۸۵۱}

\begin{document}

\date{}

\maketitle

\section{پروژهٔ نخست: سامانهٔ مزایدهٔ کالای برخط}

\subsection{توصیف پروژه}

هدف این پروژه ارائهٔ یک بستر برخط برای به مزایده گذاشتن کالا است. این بستر در قالب یک سیستم اطلاعاتی مبتنی بر وب طراحی می‌شود که در آن افراد قادر هستند کالاهای خود را با یک مبلغ پایه برای مدت محدودی به مزایده بگذارند. در ادامه افراد دارای حساب کاربری می‌توانند تا سقف اعتباری که حساب کاربری خود را شارژ کرده‌اند روی جنس قیمت بگذارند (در این صورت اعتبارشان برای خرید رزرو می‌شود). در پایان مدت زمان مزایده فردی که بالاترین قیمت را پیشنهاد کرده است به عنوان برندهٔ مزایده به فرد آگهی دهنده معرفی می‌شود و پس از تایید تحویل کالا توسط خریدار مبلغ قرارداد به حساب فروشنده واریز می‌شود.

\subsection{\lr{Economical Feasibility Analysis}}

\subsection{\lr{Schedule Feasibility Analysis}}


\section{پروژهٔ دوم: سامانهٔ قرض دادن کالای برخط}

\subsection{توصیف پروژه}

هدف این پروژه ارائهٔ یک بستر برخط برای قرض دادن و قرض گرفتن کالاست. به این شکل که در قالب یک سرویس تحت وب افراد می‌توانند اجناس خود را با تعیین مبلغ پرداختی و مبلغ بیعانه برای قرض دادن قرار می‌دهند. افرادی که حساب کاربریشان به اندازهٔ مبلغ مجموع شارژ دارد می‌توانند برای قرض گرفتند اقدام کنند. سپس مبلغ کسر می‌شود و پس از تحویل کالا پیش‌پرداخت با کسر مبلغ واسطه‌گری به حساب فرد قرض دهنده منتقل می‌شود. در انتها نیز پس از پس دادن کالا و اعلام فرد قرض دهنده باقی مبلغ به قرض دهنده منتقل شده و بیعانه خریدار نیز آزاد می‌شود.

\subsection{\lr{Economical Feasibility Analysis}}

\subsection{\lr{Schedule Feasibility Analysis}}

\section{انتخاب پروژه}

\begin{itemize}
\item
آشنایی سطحی و ناکافی دانشجویان با فیلدها و مسیرهای علمی که باعث انتخاب‌های ناصحیح و سطحی می‌شود. 
\item
نبود بستری مناسب برای شروع فعالیت و یادگیری عمیق‌تر در فیلدهای مورد علاقه بدون پرداخت هزینه‌ای زیاد.
\end{itemize}
امسال نیز انجمن عملی با هدف ایجاد چنین بستری برای دانشجویان، با بررسی نقاط ضعف و قوت برنامهٔ سال گذشته، دست به ایجاد طرحی با ساختاری نسبتا نو کرد که با نام «فیلدکس\footnote{\lr{FieldEx: Fields Experience}}» شناخته می‌شود.

\section{بخش دوم}
\subsection{پروژهٔ اول}
\begin{itemize}
\item
آشنایی عمیق‌تر مخاطبان با فیلدهای مورد علاقه‌ٔ خود در راستای انتخاب دقیق‌تر مسیر آینده‌شان
\item
کسب تجربه‌ٔ عمیق‌تر از فعالیت در یک فیلد و لمس چالش‌های مربوط با آن
\item
آشنایی بیشتر با فرآیندهای پژوهشی و حل مسئله
\item
تقویت روحیه‌ و مهارت‌های کار تیمی
\end{itemize}

\subsection{پروژهٔ دوم}
مخاطبان این طرح عمدتا از دانشجویان سال‌های اول تا سوم کارشناسی هستند. پیش‌بینی می‌شود بیشترین میزان استقبال از طرف دانشجویان سال دوم صورت بپذیرد.


\section{بخش سوم}
\begin{itemize}
\item
تیم:
گروهی از دانشجویان که با قصد فعالیت تیمی در یک فیلد به‌خصوص گرد هم جمع شده‌اند.
\item
حامی:
فردی با تجربه‌ٔ بالای کار تیمی و با دید علمی مناسب که تیم را در راستای عمل‌کرد بهتر در کار تیمی و مدیریت کارهای خود یاری می‌کند. عمدهٔ حامی‌ها از دانشجویان اواخر دوره‌ٔ کارشناسی و یا از دانشجویان ارشد انتخاب می‌شود. به عبارتی دیگر حامی تسهیل‌گر روند فعالیت تیم خواهد بود.
\item
لیدر:
لیدر به عنوان رهبر و هدایت‌گر علمی تیم راه پیش‌رو را برای تیم روشن می کند، روش کار را به دانشجویان یاد می‌دهد و در کار‌ها به آن‌ها مشورت می‌دهد. در واقع از آنجایی که دانشجویان به کار در یک حوزه علاقه دارند اما راهِ کار علمی در آن حوزه را نمی‌دانند نیاز به لیدری برای هدایت در  طول مسیر دارند. تجربه و دانش لیدر تکمیل‌کنندهٔ اشتیاق تیم و حمایت اجرایی حامی خواهد بود.
\end{itemize}

ترتیب اجرای بخش‌های مختلف رویداد به شرح زیر است:
\begin{itemize}
\item
تیم برگزاری فیلدکس در ماه اخیر چندین برنامه تحت عنوان FieldIn\footnote{\lr{Field Introduction}} برای آشنایی دانشجویان با گرایش‌ها و فیلدهای مرتبط با حوزه‌ٔ کامپیوتر و آزمایشگاه‌های مربوط به آن‌ها در دانشکده برگزار کرد. سعی بر آن بود که دانشجویان با مسائل و چالش‌های هر حوزه آشنا شده و از ظرفیت‌های آزمایشگاه‌های دانشکده مطلع گردند. همچنین این رویدادها دانشجویان را به انتخاب فیلد مورد علاقه‌ٔ خود و ثبت‌نامی آگاهانه‌تر در فیلدکس سوق می‌دهند.
\item
دانشجویان پس از اطلاع از نحوهٔ برگزاری رویداد به شناسایی افرادی با علایق مشترک با خود می‌پردازند. بناست رویدادهای فیلدین هموارکنندهٔ مسیر این شناسایی باشند. این افراد یک هفته فرصت برای اقدام به تشکیل تیم دارند.
\item
در این مرحله هر تیم برای فعالیت در یک فیلد ثبت ‌نام می‌کند. تیم برگزاری به هر تیم یک حامی با مشخصات ذکر شده نسبت می‌دهد تا اعضای آن تیم را در ادامه‌ٔ مسیر یاری کند.
\item
پس از آن تیم باید با کمک حامی به پیدا کردن لیدری به عنوان راهنمای علمی خود بپردازد. افراد تیم می‌بایست لیدر مورد نظر را متقاعد کنند که قصد فعالیتی جدی در آن حوزه را دارند و به دنبال خروجی مطلوبی از فعالیت‌های خود هستند. در صورتی که لیدر با همکاری با آن تیم موافقت کند، تیم و لیدر اقدام به تهیه‌ٔ برنامه‌ای از چگونگی فعالیت در آن فیلد می‌کنند. این برنامه باید دقیق بوده و در آن زمان‌بندی‌های تعیین شده ذکر شود. در نهایت این اطلاعات تحت قالب پروپوزالی به تیم برگزاری ارسال می‌شود.
\item
پروپوزال تهیه شده پس از بررسی توسط تیم برگزاری و با مشورت با  لیدر مربوط به آن، به تایید رسیده و یا رد می‌شود. هدف این مرحله فیلتر شدن تیم‌هایی است که به تصویر مناسب و دقیقی از ادامه‌ٔ فعالیت خود نرسیده‌اند و یا دارای جدیت کافی نیستند.
تیم‌های به تایید رسیده پس از این مرحله شروع به فعالیت خواهند کرد.
\item
هر حامی در طول فعالیت تیم مربوط، به آن تیم برای مدیریت وظایف تعیین شده و پیش‌برد کارها کمک می‌کند. از طرف دیگر لیدر نیز در طول مسیر راهنمایی‌های علمی لازم را به اعضای تیم کرده و نیز بنا به صلاحدید خود و تیم، جلساتی به طور منظم بین او و تیم برگزار‌ می‌شود تا روند اجرای کار تا آن مرحله بررسی شده و گام‌های بعدی به شکل دقیق‌تری تبیین گردند.
\item
همچنین تیم برگزاری سعی می‌کند با برگزاری رویدادهایی انگیزه‌ٔ تیم‌ها را برای پیشبرد اهداف خود حفظ کند و در صورت نیاز آن‌ها به منابع و  یا کارگاه‌های آموزشی، این امکانات را برای آنان فراهم سازد. 
\end{itemize}

زمان‌بندی اولیه‌ی برنامه به شکل زیر است:

\bgroup
\def\arraystretch{1.5}
\begin{table}[h]
\centering
\begin{tabular}{|c|c|c|}
\hline
\multicolumn{2}{|c|}{جدول زمان‌بندی} \\ \hline
آخرین رویداد FieldIn و افتتاحیه‌ی FieldEx
  & ۲ و ۳ بهمن
\\ \hline
مهلت ثبت نام تیم‌ها 
& ۱۰ بهمن
\\ \hline
مهلت ارسال پروپوزال به تیم برگزاری 
& ۲۱ بهمن
\\ \hline
پاسخ دادن تیم برگزاری و نهایی شدن تیم‌ها 
& ۲۴ بهمن
\\ \hline
اختتامیه‌ی رویداد
& ۳۱ اردیبهشت
\\ \hline
\end{tabular}
\end{table}
\egroup

\end{document}
